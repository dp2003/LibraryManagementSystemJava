\documentclass{article}
\usepackage[utf8]{inputenc}
\usepackage{graphicx}
\usepackage{geometry}
 \geometry{
 a4paper,
 total={150mm,240mm},
 left=20mm,
 top=20mm,
 }
 
\graphicspath{ {./} }

\title{Shri Vile Parle Kelavani Mandal’s\\
\textbf {SHRI BHAGUBHAI MAFATLAL POLYTECHNIC}
}

\author{}

\date{September 2020}

\begin{document}
\maketitle \LARGE
{
Program : \textbf {Information Technology}\newline 
Course Name : \textbf {Programming in Java}\newline
Course Code : \textbf {PRJ190901|}\newline
Semester : \textbf {III}\newline
Academic Term : \textbf {15th June 2020 to 7th Nov. 2020}
}
\newline
Title : \textbf{Library Management System}\\
\\
Student Name: \textbf{Durva Haresh Patel}\\
Student Roll No: \textbf{1991036}

\newpage

\begin{center}
\textbf{Index}\newline\newline
\begin{tabular}{ |c|c|c| } 
 \hline
 \textbf {Sr no.}  & \textbf{Topic }  & \textbf{Page No.} \\
 \hline
1 & Abstract  & 3  \\ 
 \hline
2 & Problem Statement and features & 3\\ 
  \hline
3 & Software Requirements & 4 \\ 
 \hline
4 & Hardware Requirements & 5 \\ 
  \hline
5 & Module Implementation & 5 \\ 
 \hline
6 & Mini Project Source Code & 6-10 \\ 
  \hline
7 & Mini Project Result & 11-16 \\ 
 \hline
8 & Conclusion & 17 \\ 
 \hline
9 & References & 17 \\ 
 \hline
\end{tabular}
\end{center}

\newpage

\begin{center}
\textbf{Abstract}
\end{center}
\Large
\par 
My Mini Project titled "Library Management System" is a software for monitoring and controlling the Transactions in a Library. This Project is designed & coded in Netbeans IDE and database Management is handled by PHP MyAdmin, a free software tool written in PHP intended to handle the administration of MySQL and MariaDB.This Software mainly focuses on basic operations like to insert new information, searching books and facility to issue book and give feedback. My Project is easy to use for both beginners ad advanced users. It features an attractive user interface which is very interactive, designed by using the concept of Java Swing available in Netbeans\newline\newline
\begin{center}
\LARGE
\textbf{Problem Statement and Features:}\newline
\end{center}
\Large
Develop a Java Mini Project for Library Management System which offers the following features:\\

\begin{itemize}  
\item Insert Data into Database.
\item View records in the Database.
\item Issue Book.
\item Search for Available Book
\item Feedback or Rate a particular Book.
\end{itemize} 

\newpage

\begin{center}
\textbf{SOFTWARE REQUIREMENTS}\newline
\end{center}
\textbf{1.	Java: JDK1.8}\newline
\includegraphics[width=5cm]{java5}
%\includegraphics[scale=1.2, angle=45]{java5} 
\newline
Fig. 1 Oracle Java Logo\newline
\par Java is a general-purpose programming language that is class-based, object-oriented, and designed to have as few implementation dependencies as possible. Java is fast, reliable and secure. From desktop to web applications, scientific supercomputers to gaming consoles, cell phones to the Internet, Java is used in every nook and corner.Not only is Java the official programming language for Android app development (along with Kotlin), Java itself is used by Google for large parts of the Android internals.\newline
\par JFreechart, JasperReport, Mail and Activations, MySQL, XAMPP, Netbeans\newline\newline
\textbf{MySQL} is easy to use. It is secure and consist of a solid data security layer ta protects sensitive data from intruders. Client Server Architecture, free to download, it is scalable ,speed and high flexibility.\newline
\textbf{XAMPP} has he ability to serve web pages on he World Wide Wed. A speial tool is provided to password protect the most important parts the package. XAMPP also provides support for creating and manipulating databases in MariaDB and SQLite among others\newline
\textbf{NetBeans IDE}
\begin{itemize}  
\item Best Support for latest Java Technologies. 
\item Fast and Smart Code Editing
\item Easy and Efficient Project Management
\item Rapid user Interface Development
\item Write Bug Free Code
\end{itemize} 

\newpage

\begin{center}
\textbf{HARDWARE REQUIREMENTS}\newline
\end{center}
\Large
\begin{itemize}  
\item Windows 7 or higher (32bit or 64bit)
\item Intel i3 processor (1.30 GHz)
\item Minimum 4GB RAM
\item 250GB Free Disk Space \newline
\end{itemize}
\Large
\begin{center}
\textbf{Module Implementation} \newline
\end{center}
\begin{center}
\large
\begin{tabular}{ |c|c|c|c|} 
 \hline
 \textbf {Sr no.}  & \textbf{Module Name }  & \textbf{Description} & \textbf{Implementation date} \\
 \hline
   &  Defining the   & Collecting all information &   \\ 
 1 &  Requirements   & needed for implementation of a  &   4-10-2020   \\
   &                 &  Library Management System   &      \\
\hline
   &  Designing on  & Basic Structure or an outline  &    \\ 
 2 &  paper         & of the whole project on    &  8-10-20   \\
   &                & paper.                     &      \\
\hline
   &  Designing the  & Design the Frames using Java Swing   &    \\ 
 3 &  Frames practically.   & on an IDE without     &  16-10-20   \\
   &                & implementation.                    &      \\
\hline
  &  Login Form    & Login Form enables the user to            &     \\
 4& implementation      & login to the Project and enables the      &  21-10-20  \\
  &                  & the connection to Database.               &       \\
 \hline
 &  Main Menu  & The Main Menu Frame contains the all      &     \\ 
 5 & implementation & the options used to access the further    &  25-10-20    \\
  &                  & Frames or options.                        &       \\
  \hline
& Insert Data  &  The Insert Data allow  the user to       &     \\
 6 & implementation  &   enter data into Database.               & 28-10 20     \\
    &                  &                                           &     \\
 \hline
 & & The issual form is used to get  &    \\ 
  &   Issual Form  & information the person who have issued &     \\
 7 &   implementation &  the form and which of book have been  & 30-10-20     \\
  &                  &  issued and store the information  &       \\
   &                 &   in the database.                        &  \\
  \hline
&   Feedback Form  & The Feedback Form all takes the &   \\ 
  8& implementation &  feedback from the user about the book   &  3-11-20  \\
  &                  &   and store it into the database. &     \\
 \hline
 &    Display Form   & The Display Form Displays the records &   \\ 
 9 & implementation &   of the existing book into the table .   & 7-11-20   \\
  &                  &                                           &     \\
  \hline
 &  Search Form  & Search Form searches the Records of &    \\ 
 10 & implementation  & the book when its name is entered .  & 12-11-20    \\
  &                  &                                           &      \\
 \hline
\end{tabular}
\end{center}

\newpage

\begin{center}
\textbf{Mini Project Source Code}\newline\newline
\end{center}
\Large
Login Form\newline\newline
\includegraphics[width=15cm]{Login Form}
\newpage
Main Menu\newline\newline
\includegraphics[width=15cm]{MainMenu}
\newpage
Insert Data\newline
\includegraphics[width=15cm]{InsertData}
Display Data\newline
\includegraphics[width=15cm]{DisplayRecords}
\newpage
Issue Form\newline\newline
\includegraphics[width=15cm]{IssueBook}
Feedback Form\newline\newline
\includegraphics[width=15cm]{FeedbackForm}
\newpage
Search Data\newline\newline
\includegraphics[width=15cm]{SearchData}
Code for getting the JFrame in the middle of the Screen\newline\newline
\includegraphics[width=15cm]{MiddleFrame}
\newpage
\begin{center}
\LARGE
\textbf{Mini Project Result}\newline
\end{center}
\Large
Login Form Result\newline
\begin{center}
\includegraphics[width=13cm]{loginResult}\newline\newline
\includegraphics[width=15cm]{login}
\end{center}
\newpage
Main Menu Result
\begin{center}
\includegraphics[width=13cm]{mainResut}
\end{center}
Insert Data Result
\begin{center}
\includegraphics[width=13cm]{insertResult}
\end{center}
\newpage
Insert Data Result\newline
\begin{center}
\includegraphics[width=13cm]{displayResult}\newline\newline
\includegraphics[width=13cm]{records}
\end{center}
\newpage
Issue Book Result\newline
\begin{center}
\includegraphics[width=13cm]{issualResult}\newline\newline
\includegraphics[width=13cm]{issue}
\end{center}
\newpage
Feedback Form Result\newline
\begin{center}
\includegraphics[width=13cm]{feedbackResult}\newline\newline
\includegraphics[width=13cm]{feedback}
\end{center}
\newpage
Search Book Result\newline
\begin{center}
\includegraphics[width=13cm]{searchResult}\newline\newline
\end{center}

\newpage

\LARGE
\textbf{Conclusion :}\newline
\Large
Through this Mini Project we learnt about the basics functionalities of Library Management System.We learnt about the concept designing of JFrames in Java, through which we designed all the frames in the Mini Project. We also learnt about addition of database connectivity in our Mini Project, We use the Login form to connect our project to database. My Mini Project is a simple and easy to understand project which could be understood by a beginner and and also a advanced user.\newline \newline
\LARGE
\textbf{References }:  
\Large
\begin{itemize}  
\item https://www.javatpoint.com/java-jdbc
\item https://www.javatpoint.com/java-swing
\item https://www.siteground.com/tutorials/phpmyadmin/
\item https://www.latex-project.org/help/documentation/
\end{itemize}
\end{document}